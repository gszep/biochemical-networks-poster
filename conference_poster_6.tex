\documentclass[a0,portrait]{a0poster}
\usepackage{multicol} % This is so we can have multiple columns of text side-by-side
\columnsep=100pt % This is the amount of white space between the columns in the poster
\columnseprule=3pt % This is the thickness of the black line between the columns in the poster

\usepackage[svgnames]{xcolor} % Specify colors by their 'svgnames', for a full list of all colors available see here: http://www.latextemplates.com/svgnames-colors
\usepackage{tcolorbox} % coloured boxing of sections
\usepackage{tikz} % manual placement of logos
\usepackage{palatino} % use the Palatino font

\usepackage{graphicx} % Required for including images
\usepackage{enumitem}
\graphicspath{{figures/}} % Location of the graphics files

\usepackage{booktabs} % Top and bottom rules for table
\usepackage[font=small,labelfont=bf]{caption} % Required for specifying captions to tables and figures

\usepackage{amsfonts, amsmath, amsthm, amssymb} % For math fonts, symbols and environments
\usepackage{wrapfig} % Allows wrapping text around tables and figures

\begin{document}

%----------------------------------------------------------------------------------------
% title and authors
\begin{minipage}[b]{0.99\linewidth}
	\begin{center}
		\veryHuge \color{DarkRed} \textbf{Reverse-engineering natural computation} \color{Black}\\ % Title
		\Huge\textit{using reaction-diffusion approaches beyond linear stability}\\[2.4cm] % Subtitle
		\huge \textbf{Gregory Sz\'ep, Luca Cardelli, Attila Csik\'asz-Nagy}\\[0.5cm]
		\vspace{1cm}\includegraphics[width=70cm]{abstract}
	\end{center}
\end{minipage}

% organisation logos
{\begin{tikzpicture}[remember picture, overlay]
     \node [anchor=north west, inner sep=6cm]  at (-8,34)
     {\includegraphics[height=7cm]{kcl}};
  \end{tikzpicture}}

{\begin{tikzpicture}[remember picture, overlay]
     \node [anchor=north east, inner sep=6cm]  at (81,34)
     {\includegraphics[height=6cm]{mrc}};
  \end{tikzpicture}}

{\begin{tikzpicture}[remember picture, overlay]
     \node [anchor=south west, inner sep=6cm]  at (58,-89)
     {\includegraphics[height=7cm]{epsrc}};
  \end{tikzpicture}}
\vspace{-3cm}

% itemized thumbs up/down icons
\newcommand*\up{\item[{\includegraphics[width=0.75em]{up}}\,\,]}
\newcommand*\down{\item[{\includegraphics[width=0.75em]{down}}\,\,]}

%----------------------------------------------------------------------------------------

\huge

\medbreak
\begin{tcolorbox}[boxrule=2pt,arc=3.4pt,boxsep=2mm]
\begin{center}\color{DarkRed}
\textbf{What is the \textit{minimal} reaction network that obeys a given \textit{response function}?}
\end{center}
\end{tcolorbox}

\begin{itemize}[leftmargin=5cm]
	\up Mappings between between algorithms and biochemical reaction networks have been\\
	explored to pave the path towards molecular programming in cells \cite{Dalchau2018ComputingClocks}
	\up Synthetic biology and in-vitro reconstructive approaches \cite{Loose2011MinMinE}
	enable scientists to probe the validity of proposed reaction networks
	\down Equally valid networks can be 
	\down No general mapping exists that takes model complexity into consideration
\end{itemize}
guides experiments
\begin{center}
\includegraphics[width=0.2\linewidth]{inference}
\end{center}\noindent
The function of known biochemical networks such as the MAPK pathway,
circadian rhythms and cell-cycles can be understood in terms of simple
response functions; decomposition of large networks into switches and
clocks are proposed in literature.
\medbreak\medbreak\noindent
The networks found in nature are far from the least complicated realisations
of particular response functions. The additional complexity can be
explained by molecular evolutionary paths towards robust biological function
\cite{DanielsSloppinessBiology}.
\medbreak
\begin{tcolorbox}[boxrule=2pt,arc=3.4pt,boxsep=2mm]
\begin{center}\color{DarkRed}
\textbf{Can model reduction methods \cite{Cardelli2016NoiseSwitches} identify relevant components,
parameters and reduce complexity in reaction networks?}
\end{center}
\end{tcolorbox}
\begin{center}
\includegraphics[width=0.9\linewidth]{reduction}
\end{center}
\begin{tcolorbox}[boxrule=2pt,arc=3.4pt,boxsep=2mm]
\begin{center}\color{DarkRed}
\textbf{Using measures of relative complexity between two given networks,
can we construct evolutionary trees and understand how primitive switches
and clocks evolved?}
\end{center}
\end{tcolorbox}

\section*{2 Patterns in dynamic populations}
Ever since Turing formulated the differential diffusion condition \cite{}
for pattern formation, whether a biological pattern is truly driven by a
diffusion instability or not has been a matter of debate and speculation.
\medbreak\medbreak\noindent
It is conceivable that the differential diffusion condition is satisfied
by the time-scale separation between cytosolic, membrane and inter-cellular
reactions.
\medbreak\medbreak\noindent
Finite element reaction-diffusion simulations can take these effects into
account explicity at an enourmous computational cost, which leads researches to
resort to more abstract Kuramoto-type models.
\medbreak
\begin{tcolorbox}[boxrule=2pt,arc=3.4pt,boxsep=2mm]
\begin{center}\color{DarkRed}
\textbf{Can we construct a computationally tractable reaction-diffusion model
that takes cell division and death into account?}
\end{center}
\end{tcolorbox}
\begin{center}
\includegraphics[width=0.9\linewidth]{population}
\end{center}
\large\begin{align*}
	\partial_t\rho=\partial_x^2(D\rho)+R(\rho)\qquad
	\partial_tD=?
\end{align*}
\normalsize
Evolution of biochemical chemical networks is not just driven by random mutation,
but are subject to environmental pressures and competition. Snythetic approaches
to bacterial competition experiments \cite{} allow us to probe the fitness of
different chemical networks.
\medbreak
\begin{tcolorbox}[boxrule=2pt,arc=3.4pt,boxsep=2mm]
\begin{center}\color{DarkRed}
\textbf{How to we model competition between cell types that contain
different reaction networks?}
\end{center}
\end{tcolorbox}
\begin{center}
\includegraphics[width=0.9\linewidth]{competition}
\end{center}
\section*{4 Outlook}
\begin{itemize}
	\item Help synthetic biologists
	\item Help systems biologists
	\item Help evolutionary biologists
	\item Do theory; be happy
\end{itemize}

\section*{3 Geometrisation approach}
Recent advances in the theory of pattern formation \cite{Halatek2018} suggest that
the dynamics of \textit{local equilibria} provide general insight into pattern
formation beyond linear stability analysis.
\begin{center}
\includegraphics[width=0.99\linewidth]{motivation}
\end{center}
Such a theory requires a descritption of the system in phase space, moving in
harmony with reactive and diffusive forces. Below I present preliminary equations
of motion:\medbreak
\large\begin{align*}
	\partial_k\left(R(k)\Gamma(k)\right)+\partial_k^2 V(k)=0\qquad\quad
	\\\\
	\Gamma(k,t):=\int_{\mathbb{R}^M}\!\delta(k-\rho(x,t))\,\mathrm{d}x\qquad\quad
	\\
	V(k,t):=D\int_{\mathbb{R}^M}\!
		(\partial_x\rho(x,t))^2\,\delta(k-\rho(x,t))\,\mathrm{d}x
\end{align*}

% %----------------------------------------------------------------------------------------
% %	CONCLUSIONS
% %----------------------------------------------------------------------------------------
%
% \color{SaddleBrown} % SaddleBrown color for the conclusions to make them stand out
%
% \section*{Conclusions}
% Despite being a petroleum- and gas-rich country, Algeria is making efforts to exploit its renewable energies. The Algerian government has adopted new renewable energy laws and financial support for the investors to facilitate the exploitation of the renewable energies for electricity production and direct utilizations. Algeria has relatively abundant geothermal resources especially in the northeastern parts but not totally used.
% \color{Black} % Set the color back to DarkSlateGray for the rest of the content
%
% %----------------------------------------------------------------------------------------
% %	FORTHCOMING RESEARCH
% %----------------------------------------------------------------------------------------
%
% \section*{Forthcoming Research}
%
% Simulation of thermodynamic properties of the thermal fluid and power output with longevity using geological, hydrogeological, and geothermal data from NE-Algerian geothermal reservoirs.
%
%  %----------------------------------------------------------------------------------------
% %	REFERENCES
% %----------------------------------------------------------------------------------------
%

\begin{minipage}[t][][b]{0.99\textwidth}
\bibliographystyle{style} % Plain referencing style
\bibliography{mendeley_v2} % Use the example bibliography file sample.bib
\end{minipage}
%----------------------------------------------------------------------------------------

\end{document}
